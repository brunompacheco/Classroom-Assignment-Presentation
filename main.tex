\documentclass{beamer}

% Pacote de estilo da UFSC
\usepackage{style/ufsc}

% Incluir arquivos da pasta figuras
\graphicspath{{./figuras/}}

% Pacote de texto aleatório
\usepackage{lipsum}


% Início do documento
\begin{document}

%%
%%	Incluir \capa para os slides
%% 
\titulo{University Timetabling}
\autores{Bruno M. Pacheco, Pedro Kretzschmar}
\universidade{Universidade Federal de Santa Catarina}
\capa


%%
%%	Chamar o ambiente frame para slides comuns
%%
\begin{frame}{Introduction}
\begin{itemize}
    \item What is university course timetabling? \begin{itemize}
        \item University course timetabling is a large allocation problem, in which both times and rooms are determined for each class meeting. Besides that, other characteristics can be added to make the problem even more complex, like room attributes, lectures that require more than a one-time slot, professor or room restrictions, and others;
    \end{itemize}
    \item Why is it important?
    \begin{itemize}
        \item The improvement of the timetable process can bring a better experience for the students and the course. Evenly distributed lectures during the week, less time wasted changing rooms, a better understanding of the use of university resources, etc.
    \end{itemize}
\end{itemize}
\end{frame}


\begin{frame}{Problem Variations}
\begin{itemize}
\item Post Enrollment Course Timetabling (PE-CTT)
    \begin{itemize}
    \item Definition: Schedule a set of events into rooms and time-slots based on the necessity created during the enrollment period of the university;
    \item Some research or state-of-the-art or review
    \end{itemize}
\item Curriculum-Based Course Timetabling (CB-CTT)
    \begin{itemize}
    \item Definition: Schedule a set of events to rooms and time-slots, respecting the curricula of the course, aiming to evenly spread the lectures in the weekdays, and preserving the same room for an event;
    \item Some research or state-of-the-art or review
    \end{itemize}
\item \textbf{Classroom assignment problem}
    \begin{itemize}
    \item Definition: Schedule a set of events into suitable rooms and time-slots. In this variant, the optimization aims to produce results that don't break the continuity of the events.
    \item Some research or state-of-the-art or review
    \end{itemize}
\end{itemize}
\end{frame}

\begin{frame}{Problem Description}
    We will tackle the \emph{Classroom Assignment Problem}, in which the faculties and departments of a university define their schedule (timetable) and the problem is to assign rooms to each lecture.

\vspace{10mm}

\begin{columns}[t]
    \begin{column}{.75\textwidth}
	\begin{table}[H]
	    \centering
	    \resizebox{\textwidth}{!}{
	    \begin{tabular}{c | c | c | c | c | c | c}
		Time & Monday & Tuesday & Wednesday & Thursday & Friday & Saturday \\
		\hline
		7:30 & Course 1 & Course 2 &  &  & Course 1 // Course 3 &  \\
		8:20 & Course 1 // Course 5 & Course 2 // Course 6 &  & Course 3 // Course 7 & Course 1 // Course 3 & \\
		9:10 & Course 1 // Course 5 & Course 6 &  & Course 3 // Course 7 &  & \\
		10:10 & Course 4 &  &  & Course 3 // Course 5 &  & \\
		11:00 & Course 4 &  &  & Course 3 // Course 5 &  & \\
	    \end{tabular}}
	\end{table}
    \end{column}

    \begin{column}{.25\textwidth}
	\begin{table}[H]
	    \centering
	    \resizebox{.9\textwidth}{!}{
	    \begin{tabular}{ c | c}
	    Course & Assigned Room \\
	    \hline
	    Course 1 & Room A \\
	    Course 2 & Room B \\
	    Course 3 & Room B \\
	    Course 4 & Room A \\
	    Course 5 & Room C \\
	    Course 6 & Room A \\
	    Course 7 & Room A \\
	\end{tabular}}
	\end{table}
    \end{column}
\end{columns}
\end{frame}


\begin{frame}{Features}
    There are five main variables in this problem:
\begin{description}
    \item[Lecture] It varies from the common interpretation in that it occupies a room for a single time period.
    \item[Course] Has one or more lectures and defines the \emph{size} and \emph{attributes} needed from the room.
    \item[Room] Has size and attributes (e.g., computers, beamers, lab).
    \item[Time period] A time interval in which the lectures occur.
    \item[Pattern] A set of lectures from a given course that will take place in the same room.
\end{description}
\end{frame}

\begin{frame}{Features - Example}
    An instance of the problem must provide information on the courses and the rooms on the time periods.
    \begin{table}[H]
        \centering
	\resizebox{.9\textwidth}{!}{
        \begin{tabular}{c | c | c | c | c }
	    Course ($c$) & Size ($size_c$) & Attributes ($att_c$) & Lectures ($l^{(c)}$) & Time periods ($t$) \\
	    \hline
	    $c_1$ & 125 & - & $l_1^{(1)}$ & $t_1$ \\
		  &  &  & $l_2^{(1)}$ & $t_2$ \\
		  &  &  & $l_3^{(1)}$ & $t_5$ \\
	    \hline
	    $c_2$ & 25 & computers, beamer & $l_1^{(2)}$ & $t_2$ \\
		  &  &  & $l_2^{(2)}$ & $t_3$ \\
		  &  &  & $l_3^{(2)}$ & $t_4$ \\
	\end{tabular}}
    \end{table}

    \begin{table}[H]
        \centering
	\resizebox{.9\textwidth}{!}{
        \begin{tabular}{c | c | c | c}
	    Room ($r$) & Size ($size_r$) & Attributes ($att_r$) & Available time periods ($T_r$) \\
	    \hline
	    $r_1$ & 150 & beamer & $t_1, t_2, t_3, t_4$ \\
	    $r_2$ & 300 & computer, beamer & $t_1, t_2, t_3, t_4, t_5$
	\end{tabular}}
    \end{table}
\end{frame}

\begin{frame}{Features - Example}
    Preprocessing the input gets us the patterns for each course and the feasible rooms.
    \begin{table}[H]
        \centering
	\resizebox{.9\textwidth}{!}{
        \begin{tabular}{c | c | c | c | c | c | c | c}
	    Course ($c$) & $t_1$ & $t_2$ & $t_3$ & $t_4$ & $t_5$ & Feasible Rooms ($R_c$) & Patterns ($P_c$) \\
	    \hline
	    $c_1$ & $l_1^{(1)}$ & $l_2^{(1)}$ &  &  & $l_3^{(1)}$ & $r_2 $ & $\left\{ l_1^{(1)} \right\}, \left\{ l_2^{(1)} \right\} , \left\{ l_3^{(1)} \right\}, \left\{ l_1^{(1)},l_2^{(1)} \right\} ,  $ \\
		  & & & & & & & $\left\{ l_1^{(1)}, l_3^{(1)} \right\} ,\left\{ l_2^{(1)},l_3^{(1)} \right\} , \left\{ l_1^{(1)},l_2^{(1)}, l_3^{(1)} \right\}$ \\
		  \hline
	    $c_2$ & & $l_1^{(2)}$ & $l_2^{(2)}$ & $l_3^{(2)}$ & & $r_1, r_2 $ & $\left\{ l_1^{(2)} \right\}, \left\{ l_2^{(2)} \right\} , \left\{ l_3^{(2)} \right\}, \left\{ l_1^{(2)},l_2^{(2)} \right\} ,  $ \\
		  & & & & & & & $\left\{ l_1^{(2)}, l_3^{(2)} \right\} ,\left\{ l_2^{(2)},l_3^{(2)} \right\} , \left\{ l_1^{(2)},l_2^{(2)}, l_3^{(2)} \right\}$ \\
	\end{tabular}}
    \end{table}
    For this instance, a possible solution can be
    \begin{table}[H]
        \centering
	\resizebox{.4\textwidth}{!}{
        \begin{tabular}{c | c | c}
	    Course ($c$) & Pattern ($p$) & Room ($r$) \\
	    \hline
	    $c_1$ & $\left\{ l_1^{(1)} \right\} $ & $r_2$ \\
		  & $\left\{ l_2^{(1)} \right\} $ & $r_1$ \\
		  & $\left\{ l_3^{(1)} \right\} $ & $r_2$ \\
	    \hline
	    $c_2$ & $\left\{ l_1^{(2)}, l_2^{(2)}, l_3^{(2)} \right\} $ & $r_2$
	\end{tabular}}
    \end{table}
\end{frame}

\begin{frame}{Problem Formulation}
    First, the definition and notation we use for the formulation.

    \begin{description}
	\item[$C$] is the set of all courses $c$.
	\item[$L_c$] is the set of all lectures that belong to $c$, that is, $L_c = \left\{ l_1^{(c)}, l_2^{(c)},\ldots \right\} $. Also, $L = \dot{\bigcup}_{c\in C}  L_c$.
	\item[$P_c$] is the set of all patterns $p$ of $c$, that is, the power set of $L_c$. Again, $P = \dot{\bigcup}_{c\in C}  P_c$. We also define $P_l$ as the set of all patterns that contain lecture $l$.
	\item[$R$] is the set of all rooms $r$. $R_c$ is the set of feasible rooms for course $c$ and, in a similar fashion, $R_p$ is the set of feasible rooms for a pattern $p$.
	\item[$T$] is the set of all time periods $t$. $T_p$ is the set of time periods for each lecture of pattern $p$. $T_r$ is the set of time periods in which room $r$ is available.
    \end{description}
\end{frame}

\begin{frame}{Problem Formulation}
    Some useful attributes are
    \begin{description}
	\item[$length_{c|p} $] is the number of lectures in a given course or pattern.
	\item[$att_{c|r}$] are the attributes required/provided by a course/room.
	\item[$size_{c|r}$] is the size of a course or a room.
    \end{description}
\end{frame}

\begin{frame}{Problem Formulation}
    The binary variables $x_{p,r}$ represent the assignment of room $r$ for all the lectures in pattern $p$.

    The measure of a given assignment will be modeled by $w_{p,r}$. This way, the quality of a solution is \[
    \sum_{p \in P} \sum_{r\in R} w_{p,r}x_{p,r}
    ,\] which we aim to maximize.
\end{frame}

\begin{frame}{Problem Formulation}
    The first constraint is that no more than one lecture is assigned to each room in each time period \[
    \sum_{p \in P_{r,t}} x_{p,r} \le 1,\, r\in R,\,t\in T_r
    ,\] where $P_{r,t}$ is the set of patterns that contain a lecture in time $t$ for which room $r$ is suitable, i.e, $P_{r,t} = \left\{ p \in P : r \in R_p,\,t\in T_p \right\} $.
\end{frame}

\begin{frame}{Problem Formulation}
    From our previous example
    \begin{table}[H]
        \centering
	\resizebox{.9\textwidth}{!}{
        \begin{tabular}{c | c | c | c | c | c | c | c}
	    Course ($c$) & $t_1$ & $t_2$ & $t_3$ & $t_4$ & $t_5$ & Feasible Rooms ($R_c$) & Patterns ($P_c$) \\
	    \hline
	    $c_1$ & $l_1^{(1)}$ & $l_2^{(1)}$ &  &  & $l_3^{(1)}$ & $r_2 $ & $\left\{ l_1^{(1)} \right\}, \left\{ l_2^{(1)} \right\} , \left\{ l_3^{(1)} \right\}, \left\{ l_1^{(1)},l_2^{(1)} \right\} ,  $ \\
		  & & & & & & & $\left\{ l_1^{(1)}, l_3^{(1)} \right\} ,\left\{ l_2^{(1)},l_3^{(1)} \right\} , \left\{ l_1^{(1)},l_2^{(1)}, l_3^{(1)} \right\}$ \\
		  \hline
	    $c_2$ & & $l_1^{(2)}$ & $l_2^{(2)}$ & $l_3^{(2)}$ & & $r_1, r_2 $ & $\left\{ l_1^{(2)} \right\}, \left\{ l_2^{(2)} \right\} , \left\{ l_3^{(2)} \right\}, \left\{ l_1^{(2)},l_2^{(2)} \right\} ,  $ \\
		  & & & & & & & $\left\{ l_1^{(2)}, l_3^{(2)} \right\} ,\left\{ l_2^{(2)},l_3^{(2)} \right\} , \left\{ l_1^{(2)},l_2^{(2)}, l_3^{(2)} \right\}$ \\
	\end{tabular}}
    \end{table}
    if we had assigned room $r_2$ to both patterns $p_4^{(1)}=\left\{  l_1^{(1)}, l_2^{(1)}\right\} $ and $p_7^{(2)}=\left\{ l_1^{(2)}, l_2^{(2)}, l_3^{(2)} \right\} $, then, for $t_2$ we would have \[
    \sum_{p \in P_{r_2,t_2}} x_{p,r_2} = x_{p_4^{(1)}, r_2} + x_{p_7^{(2)},r_2} = 2
    ,\] a violation of the constraint.
\end{frame}

\begin{frame}{Problem Formulation}
    We also must ensure that no more than one room is assigned to each lecture, thus \[
	\sum_{p \in P_l} \sum_{r \in R_p} x_{p,r} \le 1,\, l \in L
    .\] 
\end{frame}

\begin{frame}{Problem Formulation}
    Finally, at most one pattern per course should be assigned to a given room, so \[
	\sum_{p \in P_c} x_{p,r} \le 1,\, c \in C,\, r \in R_c
    .\] This implies that, if two patterns from a given course are to be assigned to a single room, the "bigger" pattern must be used, so instead of \[
    x_{\left\{ l_1^{(2)}\right\} ,r_2 } = x_{\left\{ l_2^{(2)}\right\} ,r_2 } = x_{\left\{ l_3^{(2)}\right\} ,r_2 } = 1
    ,\] the solution should be only \[
    x_{\left\{ l_1^{(2)}, l_2^{(2)}, l_3^{(2)}\right\} ,r_2 } = 1
    .\] 
\end{frame}

\begin{frame}{Problem Formulation}
    With these, our formulation becomes
    \begin{align*}
        \max_{x_{p,r}} \quad & \sum_{p \in P} \sum_{r\in R} w_{p,r}x_{p,r} \\
	\textrm{s.t.} \quad & \sum_{p \in P_{r,t}} x_{p,r} \le 1, & r\in R,\,t\in T_r \\
			    & \sum_{p \in P_l} \sum_{r \in R_p} x_{p,r} \le 1, & l \in L \\
			    & \sum_{p \in P_c} x_{p,r} \le 1, & c \in C,\, r \in R_c \\
			    & x_{p,r} \in \left\{ 0,1 \right\}, & p \in P,\, r\in R_p
    .\end{align*}
\end{frame}


\begin{frame}{Conclusion}
    
\end{frame}

\contato{E-mail: \\ \url{{ mpacheco.bruno,  pedro_kretz}@gmail.com}}
\capadetras{Thank you!}

%%
%%	Remover o número de páginas do slide com [plain]
%%
\begin{frame}[plain]{Sem número de páginas}
\lipsum[1]
\end{frame}



%%
%%	Você pode usar mais de uma coluna! Inclusão automatizada de imagens com \imagem{arquivo}{label}{legenda}
%%
\begin{frame}{Texto em duas colunas}
  \begin{columns}[c]
    \begin{column}{.5\textwidth}

	O brasão da UFSC pode ser observado na \figurename~\ref{fig:ufsc}.

    \end{column}
    \begin{column}{.5\textwidth}

	\imagem{UFSC.pdf}{fig:ufsc}{Brasão da UFSC.}

    \end{column}
  \end{columns}
\end{frame}


%%
%%	Caixas de texto
%%
\begin{frame}{Caixas de Texto}

\begin{block}{Bloco Normal}
Conteúdo do bloco normal.
\end{block}

\begin{alertblock}{Bloco de Alerta}
Conteúdo do bloco de alerta.
\end{alertblock}

\begin{exampleblock}{Bloco de Exemplo}
Conteúdo do bloco de exemplo
\end{exampleblock}

\end{frame}


%%
%%	Equações
%%
\begin{frame}{Equações}

Observe a Equação \ref{eq:sum}.

% Iniciar ambiente equation
\begin{equation}\label{eq:sum}
\sum_{n=1}^\infty \frac{1}{n^2} = \lim_{n \to \infty} \left( \frac{1}{1^2} + \frac{1}{2^2} + \cdots + \frac{1}{n^2} \right) = \frac{\pi^2}{6}
\end{equation}

Observe as Equações \ref{eq:bhaskara}.

% Iniciar ambiente split dentro de equation para várias linhas
\begin{equation}\label{eq:bhaskara}
\begin{split}
x_1 = \frac{-b + \sqrt{b^2 - 4ac}}{2a} \\
x_2 = \frac{-b - \sqrt{b^2 - 4ac}}{2a}
\end{split}
\end{equation}

\end{frame}


%%
%%	Incluir \capadetras para agradecimentos e contato.
%%
\contato{Contato: \\ \url{{autor1, autor2, autor3}@ufsc.br}}
\capadetras{Thank you!}

\end{document}
