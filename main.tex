\documentclass{beamer}

% Pacote de estilo da UFSC
\usepackage{style/ufsc}

% Incluir arquivos da pasta figuras
\graphicspath{{./figuras/}}

% Pacote de texto aleatório
\usepackage{lipsum}


% Início do documento
\begin{document}

%%
%%	Incluir \capa para os slides
%% 
\titulo{Apresentação em \LaTeX!}
\autores{Autor 1, Autor 2 e Autor 3}
\universidade{Universidade Federal de Santa Catarina}
\capa


%%
%%	Chamar o ambiente frame para slides comuns
%%
\begin{frame}{Introdução}
\lipsum[75]
\end{frame}


%%
%%	Remover o número de páginas do slide com [plain]
%%
\begin{frame}[plain]{Sem número de páginas}
\lipsum[1]
\end{frame}



%%
%%	Você pode usar mais de uma coluna! Inclusão automatizada de imagens com \imagem{arquivo}{label}{legenda}
%%
\begin{frame}{Texto em duas colunas}
  \begin{columns}[c]
    \begin{column}{.5\textwidth}

	O brasão da UFSC pode ser observado na \figurename~\ref{fig:ufsc}.

    \end{column}
    \begin{column}{.5\textwidth}

	\imagem{UFSC.pdf}{fig:ufsc}{Brasão da UFSC.}

    \end{column}
  \end{columns}
\end{frame}


%%
%%	Caixas de texto
%%
\begin{frame}{Caixas de Texto}

\begin{block}{Bloco Normal}
Conteúdo do bloco normal.
\end{block}

\begin{alertblock}{Bloco de Alerta}
Conteúdo do bloco de alerta.
\end{alertblock}

\begin{exampleblock}{Bloco de Exemplo}
Conteúdo do bloco de exemplo
\end{exampleblock}

\end{frame}


%%
%%	Equações
%%
\begin{frame}{Equações}

Observe a Equação \ref{eq:sum}.

% Iniciar ambiente equation
\begin{equation}\label{eq:sum}
\sum_{n=1}^\infty \frac{1}{n^2} = \lim_{n \to \infty} \left( \frac{1}{1^2} + \frac{1}{2^2} + \cdots + \frac{1}{n^2} \right) = \frac{\pi^2}{6}
\end{equation}

Observe as Equações \ref{eq:bhaskara}.

% Iniciar ambiente split dentro de equation para várias linhas
\begin{equation}\label{eq:bhaskara}
\begin{split}
x_1 = \frac{-b + \sqrt{b^2 - 4ac}}{2a} \\
x_2 = \frac{-b - \sqrt{b^2 - 4ac}}{2a}
\end{split}
\end{equation}

\end{frame}


%%
%%	Incluir \capadetras para agradecimentos e contato.
%%
\contato{Contato: \\ \url{{autor1, autor2, autor3}@ufsc.br}}
\capadetras{Obrigado!}

\end{document}