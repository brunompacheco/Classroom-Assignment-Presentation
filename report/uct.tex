\section{University Course Timetabling}

The University Course Timetabling, in its most broad description, is an allocation problem in which both time-slots and rooms are fitted for each class meeting (lecture). This is a very challenging problem that involves solving issues like selection of the room with the right attributes to the lecture, allocation of the lectures in more than one time-slot if required without breaking its continuity, arranging the lectures respecting the professors' time restriction, and others.


Students, professors, and the university's staff can have a better experience at the university with an improved timetable process. A better experience can be achieved through:
\begin{itemize}
    \item Evenly distributed lectures during the week;
    \item Less time wasted changing between rooms;
    \item Better understanding of the use of university resources;
    \item Less time wasted allocating the lectures manually.
\end{itemize}

This problem can be approached in different ways. During the lecture three main variations were briefly explained: the Post Enrollment Course Timetabling (PE-CTT), the Curriculum-Based Course Timetabling (CB-CTT), and the Classroom Assignment Problem.

The PE-CTT problem focus on scheduling a set of events into rooms and time-slots based on the necessity created during the enrollment period of the university. The constraints of this problem are all related to events, penalizing late, consecutive, and isolated ones

The CB-CTT problem is related to schedule a set of events to rooms and time-slots, respecting the curricula of the courses, aiming to give the students more comfort during their semester. The solution aims to spread the lectures evenly during the week, preserve the same room. In this variation, the constratins mainly involve curricula and courses, targeting the features said above.

Finally, the  Classroom Assignment Problem consists of scheduling a set of events with already defined time-slots into suitable rooms. Here, the optimization aims to produce results that do not break the continuity of the events, at the same time that handles other common restrictions, as mentioned above.

