\documentclass[a4paper,11pt]{article}
\usepackage[T1]{fontenc}		% Seleção de códigos de fonte
\usepackage[utf8]{inputenc}		% Codificação do documento (conversão
\usepackage{indentfirst}		% Indenta o primeiro parágrafo de cada seção.
\usepackage{graphicx}			% Inclusão de gráficos
\usepackage{subcaption}				% enables the use of subfigures in floats
\usepackage[
%	a4paper,
left=2cm,
right=1.5cm,
bottom=1.5cm,
top = 2.5cm, 
foot=0.7cm]{geometry}
\usepackage{url}
\usepackage{setspace}
\usepackage{amsmath}
\usepackage{amsfonts}
\usepackage{fancyhdr}
\usepackage{multirow}
\usepackage{tabularx}
\usepackage{placeins}
\usepackage{natbib}
\usepackage{import}
\usepackage[nottoc]{tocbibind} % insere as referências no sumário
\usepackage{babel}
\usepackage{hyperref}
\usepackage{float}

\pagestyle{fancy}
\fancyhf{}
\lhead{DAS410049 - Integer Programming}
\rhead{Final Report}
\rfoot{\thepage}

\begin{document}
	\thispagestyle{empty}
\begin{center}
	\includegraphics[height=2cm]{imagens/logoUFSCsimples.png} \\
	{\Large Universidade Federal de Santa Catarina -- UFSC} \\
	{\Large Centro Tecnológico -- CTC} \\
	{\Large Departamento de Automação e Sistemas -- DAS} \\
	\vspace{1cm}
	{\large DAS410049 - Integer Programming} \\
	\vfill
	\large{\textbf{University Course Timetabling - Classroom Assignment} \\
	} 
	\vspace{1cm}
	% Integrantes: \\
    Bruno Machado Pacheco (202100361) \\
    Pedro Kretzschmar (XXXXXXXXX) \\
    \vfill
	Florianópolis, \today.
\end{center}

\clearpage

\tableofcontents

\clearpage

%%%%%%%%%%%%%%%%%%%%%%%%%%%%%%%%%%%%%%%%%%%%%%%%%%%%%%%%%

\section{Introduction}

This report has the objective to explain the contents of the lecture on University Course timetabling (UCT). The content presented here are the importance of the subject to society, it's variations, the formulation of the problem in focus and a use case. The content of the lecture was based in the paper Integer Programming Methods for Large Scale Practical
2 Classroom Assignment Problems by Phillips et al.


\FloatBarrier
The present report has the objective to explain the contents of the lecture about university course timetabling (UCT). The content that will be explained here are the important of the subject to society, it's variations, the formulation of the problem, an use case using the formulation described, and finally the conclusion.
\newpage

\section{University Course Timetabling}

The University Course Timetabling, in its most broad description, is an allocation problem in which both time-slots and rooms are fitted for each class meeting (lecture). This is a very challenging problem that involves solving issues like selection of the room with the right attributes to the lecture, allocation of the lectures in more than one time-slot if required without breaking its continuity, arranging the lectures respecting the professors' time restriction, and others.


Students, professors, and the university's staff can have a better experience at the university with an improved timetable process. A better experience can be achieved through:
\begin{itemize}
    \item Evenly distributed lectures during the week;
    \item Less time wasted changing between rooms;
    \item Better understanding of the use of university resources;
    \item Less time wasted allocating the lectures manually.
\end{itemize}

This problem can be approached in different ways. During the lecture three main variations were briefly explained: the Post Enrollment Course Timetabling (PE-CTT), the Curriculum-Based Course Timetabling (CB-CTT), and the Classroom Assignment Problem.

The PE-CTT problem focus on scheduling a set of events into rooms and time-slots based on the necessity created during the enrollment period of the university. The constraints of this problem are all related to events, penalizing late, consecutive, and isolated ones

The CB-CTT problem is related to schedule a set of events to rooms and time-slots, respecting the curricula of the courses, aiming to give the students more comfort during their semester. The solution aims to spread the lectures evenly during the week, preserve the same room. In this variation, the constratins mainly involve curricula and courses, targeting the features said above.

Finally, the  Classroom Assignment Problem consists of scheduling a set of events with already defined time-slots into suitable rooms. Here, the optimization aims to produce results that do not break the continuity of the events, at the same time that handles other common restrictions, as mentioned above.


\FloatBarrier
The university course timetabling is an allocation problem, in which both time-slots and rooms are fitted for each class meeting (lecture). This is a very challenging problem that involves solving issues like, select the room with the right attributes to the lecture, allocate the lectures in more than one time-slot if required, without breaking its continuity, arranging the lectures respecting the professors' time restriction, and others.


Students, professors, and the university's staff can have a better experience at the university with an improved timetable process. A better experience can be achieved through:
\begin{itemize}
    \item Evenly distributed lectures during the week;
    \item Less time wasted changing between rooms;
    \item Better understanding of the use of university resources;
    \item Less time wasted allocating the lectures manually.
\end{itemize}

This problem can be approached in different ways. During the lecture three main variations were briefly explained, the Post Enrollment Course Timetabling (PE-CTT), the Curriculum-Based Course Timetabling (CB-CTT), and the Classroom Assignment Problem.

The PE-CTT problem focus on scheduling a set of events into rooms and time-slots based on necessity created during the enrollment period of the university.

The CB-CTT problem is related to scheduling a set of events to rooms and time-slots, respecting the curricula of the course, aiming to give the students more comfort during their semester. The solution aims to spread the lectures evenly during the week, preserve the same room

Finally, the  Classroom Assignment Problem consists of scheduling a set of events with already defined time-slots into suitable rooms. Here, the optimization aims to produce results that don't break the continuity of the events, at the same time that handles other common restrictions, commented above.


\newpage

\section{Problem Formulation}

The formulation presented was based on 5 basic parameters directly related to the features of the problem. These were:
\begin{description}
	\item[$C$] the set of all courses $c$;
	\item[$L_c$] the set of all lectures of a course $c$;
	\item[$P_c$] the set of all patterns $p$ of $c$;
	\item[$R$] the set of all rooms $r$;
	\item[$T$] the set of all time periods $t$.
\end{description}

Then, as our working variable, we presented $\{x_{p,r}\} $, binary variables that represent the assignment of a room $r$ to a pattern $p$. From these, one can measure the quality of a solution of the problem through the quality of each room-to-pattern assignment $w_{p,r}$.

Finally, the formulation presented was
\begin{align*}
    \max_{x_{p,r}} \quad & \sum_{p \in P} \sum_{r\in R} w_{p,r}x_{p,r} \\
    \textrm{s.t.} \quad & \sum_{p \in P_{r,t}} x_{p,r} \le 1, & r\in R,\,t\in T_r  & \tag{1} \\
			& \sum_{p \in P_l} \sum_{r \in R_p} x_{p,r} \le 1, & l \in L  & \tag{2} \\
			& \sum_{p \in P_c} x_{p,r} \le 1, & c \in C,\, r \in R_c  & \tag{3} \\
			& x_{p,r} \in \left\{ 0,1 \right\}, & p \in P,\, r\in R_p
.\end{align*}

\subsection{Constraints}

As noted, the formulation presented has 3 constraints. $(1)$ limits the solution space to just those in which the rooms are assignment to no more than one pattern at each time period. Note that $P_{r,t}$ is used to indicate the set of patterns that have lectures in time $t$ for which room $r$ is suitable.

Similarly, $(2)$ ensures that feasible solutions assign no more than one room to each lecture $l$ by summing the assignments to the patterns that contain the given lecture. There, $P_l$ indicates the set of patterns that contain lecture $l$ and $R_p$ the set of rooms suitable for all lectures in pattern $p$.

Lastly, $(3)$ limits the solutions to those that assign a room to no more than one pattern for each course. This is so because $P_c$ is closed to union of its members, that is, $p_1, p_2 \in P_c \implies p_1 \cup p_2 \in P_c$. Therefore, in practice, this constraint does not limit the solution space, just the "duplicate" solutions.

\subsection{Quality Measures}

An hierarchical approach to solve the problem was presented, in which multiple quality measures are used and sorted by importance. The steps are:
\begin{enumerate}
    \item Pick the most important quality measure;
    \item Solve the problem;
    \item Add the optimal value as a constraint;
    \item Repeat.
\end{enumerate}

Even though quality measures are very specific to the instance of the problem, some of the more commonly found were presented. These are:
\begin{description}
    \item[Event hours (EH)] If a feasible solution for all lectures does not exist, measures the amount of events that got a room;
    \item[Seated student hours (SH)] Weights EH by their number of students enrolled;
    \item[Seat utilisation (SU)] Weights EH by the occupation rate of the room by the event;
    \item[Room preference (RP)] Quantifies the preference of rooms by the courses, e.g., the lecturer's preference, the distance of the room to the department's office;
    \item[Course room stability (RS)] Measures the amount of \emph{different} rooms assigned to each course;
    \item[Spare seat robustness (SR)] If the number of enrolled students is defined after the room assignment, penalises situations in which the room has few seats left.
\end{description}


\FloatBarrier
\newpage

\section{Complexity}

A brief analysis of the complexity of the problem was presented through different variations of the formulation exposed in which the constraints were relaxed or removed.

The first form presented, and most elementary one, was that in which subsequent lectures of the same course do not need the same room. This means that we can solve the assignment for each time period and, thus, do not need to assign rooms to patterns, but directly to lectures. This also means that $(3)$ is not necessary anymore. The two remaining constraints define a totally unimodular matrix, that is, the problem can be solved through its LP relaxation. Furthermore, the problem can be solved through assignment algorithms in polynomial time.

If contiguous room stability (subsequent lectures from the same course must take place in the same room) is not neglected, that is, if the formulation is just as presented, the assumptions beforehand presented are not valid and the LP relaxation may return fractional solutions. It is known, though, that fractional solutions are quite rare in real world instances of the problem.

If even more that contiguous room stability, course room stability is imposed, that is, that all lectures of a course are required to happen in the same room, then it was presented that problem instances are not even always feasible.

Finally, the complexity of the hierarchical optimisation procedure was presented. It was shown that the addition of these lexicographic constraints do not affect the integer property of the solution space in one way or the other.



\FloatBarrier
\newpage

\section{Use Case}

To demonstrate the applications of the presented problem, a real-life use case using data from the technological center from our university was presented. The data used contains information about each room available and the courses' lectures and their sizes.

To avoid diving into exceptions and particularities, some limitations were imposed by the presenters in the scope of the problem. These were
\begin{itemize}
    \item No laboratory room nor any laboratory-requiring courses were included;
    \item All ARQ**** courses were excluded with the rooms used by them;
    \item Limited course-class hierarchy support;
    \item Lectures bigger than the rooms in which they take place were downsized.
\end{itemize}

Besides these limitations, some further assumptions were required, namely
\begin{enumerate}
    \item The quality measure used was the number of lectures in each pattern;
    \item The room requirement was pattern-based instead of course-based as different lectures from the same course have different number of students enrolled;
    \item All rooms are available full time (i.e., $T_r$ is not necessary);
    \item An infeasibility constraint was added for modeling;
\end{enumerate}
which resulted in the following formulation for the use case, with the differences from the original formulation highlighted:
\begin{align*}
    \max_{x_{p,r}} \quad & \sum_{p \in P} \sum_{r\in R} \boldsymbol{w_{p}}x_{p,r} \\
    \textrm{s.t.} \quad & \sum_{p \in P_{r,t}} x_{p,r} \le 1, & r\in R,\,t\in \boldsymbol{T} \\
			& \sum_{p \in P_l} \sum_{r \in R_p} x_{p,r} \le 1, & l \in L \\
			& \sum_{p \in P_c} x_{p,r} \le 1, & c \in C,\, r \in R_c \\
			& x_{p,r} \in \left\{ 0,1 \right\}, & p \in P,\, r\in \boldsymbol{R} \\
			& \boldsymbol{x_{p,r}=0,} & \boldsymbol{p \in P,\, r \notin R_p} \\
.\end{align*}

\subsection{Implementation}

All the steps performed from the data preprocessing to the results were presented, that is,
\begin{enumerate}
    \item Preprocessing the data with Python data science suite;
    \item AMPL model development;
    \item Testing through small examples;
    \item Use of AMPL's Python API (AMPLPy) to bridge the gap;
    \item Solve using Gurobi at NEOS Server.
\end{enumerate}

Further details of the implementation can be seen in the project's repository\footnote{\href{https://github.com/brunompacheco/Classroom-Assignment-Presentation}{https://github.com/brunompacheco/Classroom-Assignment-Presentation}}.

\subsection{Results}

The instance of the problem from the use case had 19.448 constraints with 43.132 variables, which was solved by the server, after some pruning, in 7166 simples iterations. This shows the size of the problem, even with all the limitations imposed. All lectures had a room assigned to, which was expected since a feasible solution is known to exist even for the problem without the caveats imposed by the presenters.

The major challenges presented were the poor quality of the data and the lack of information about the rooms and the groupings of lectures within a given course, besides the absence of the preference for the rooms.


\FloatBarrier
\newpage

\section{Conclusion}

The lecture was a brief explanation using a complex modeling approach for the timetabling problem. To solve the UFSC-CTC timetabling problem was quite difficult [TODO: difficult like for the computer? for us?], even with a simpler version of the real problem. Other constraints could be added to make it closer to the real problem, like professors' schedules, group together classes from the same curricula, adding different classes of lectures (practical, seminars). In the end, the main goal of the lecture was to transmit how important the IP is and give a brief showcase of how it is inserted into our daily lives.

The present report had the objective to explain briefly the contents of the UCT lecture. For a deeper explanation, the slides of the lecture and the references at the project proposal are strongly recommended.
The lecture was a brief explanation using a complex modeling approach for the timetabling problem. To solve the UFSC-CTC timetabling problem was quite difficult, even with a simpler version of the real problem. Other constraints could be added to make it closer to the real problem, like professors' schedules, group together classes from the same curricula, adding different classes of lectures (practical, seminars). In the end, the main goal of the lecture was to transmit how important the IP is and give a brief showcase of how it is inserted into our daily lives.
\newpage

\end{document}
